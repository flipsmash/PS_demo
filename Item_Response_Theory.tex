\documentclass[11pt]{article}

    \usepackage[breakable]{tcolorbox}
    \usepackage{parskip} % Stop auto-indenting (to mimic markdown behaviour)
    

    % Basic figure setup, for now with no caption control since it's done
    % automatically by Pandoc (which extracts ![](path) syntax from Markdown).
    \usepackage{graphicx}
    % Maintain compatibility with old templates. Remove in nbconvert 6.0
    \let\Oldincludegraphics\includegraphics
    % Ensure that by default, figures have no caption (until we provide a
    % proper Figure object with a Caption API and a way to capture that
    % in the conversion process - todo).
    \usepackage{caption}
    \DeclareCaptionFormat{nocaption}{}
    \captionsetup{format=nocaption,aboveskip=0pt,belowskip=0pt}

    \usepackage{float}
    \floatplacement{figure}{H} % forces figures to be placed at the correct location
    \usepackage{xcolor} % Allow colors to be defined
    \usepackage{enumerate} % Needed for markdown enumerations to work
    \usepackage{geometry} % Used to adjust the document margins
    \usepackage{amsmath} % Equations
    \usepackage{amssymb} % Equations
    \usepackage{textcomp} % defines textquotesingle
    % Hack from http://tex.stackexchange.com/a/47451/13684:
    \AtBeginDocument{%
        \def\PYZsq{\textquotesingle}% Upright quotes in Pygmentized code
    }
    \usepackage{upquote} % Upright quotes for verbatim code
    \usepackage{eurosym} % defines \euro

    \usepackage{iftex}
    \ifPDFTeX
        \usepackage[T1]{fontenc}
        \IfFileExists{alphabeta.sty}{
              \usepackage{alphabeta}
          }{
              \usepackage[mathletters]{ucs}
              \usepackage[utf8x]{inputenc}
          }
    \else
        \usepackage{fontspec}
        \usepackage{unicode-math}
    \fi

    \usepackage{fancyvrb} % verbatim replacement that allows latex
    \usepackage{grffile} % extends the file name processing of package graphics
                         % to support a larger range
    \makeatletter % fix for old versions of grffile with XeLaTeX
    \@ifpackagelater{grffile}{2019/11/01}
    {
      % Do nothing on new versions
    }
    {
      \def\Gread@@xetex#1{%
        \IfFileExists{"\Gin@base".bb}%
        {\Gread@eps{\Gin@base.bb}}%
        {\Gread@@xetex@aux#1}%
      }
    }
    \makeatother
    \usepackage[Export]{adjustbox} % Used to constrain images to a maximum size
    \adjustboxset{max size={0.9\linewidth}{0.9\paperheight}}

    % The hyperref package gives us a pdf with properly built
    % internal navigation ('pdf bookmarks' for the table of contents,
    % internal cross-reference links, web links for URLs, etc.)
    \usepackage{hyperref}
    % The default LaTeX title has an obnoxious amount of whitespace. By default,
    % titling removes some of it. It also provides customization options.
    \usepackage{titling}
    \usepackage{longtable} % longtable support required by pandoc >1.10
    \usepackage{booktabs}  % table support for pandoc > 1.12.2
    \usepackage{array}     % table support for pandoc >= 2.11.3
    \usepackage{calc}      % table minipage width calculation for pandoc >= 2.11.1
    \usepackage[inline]{enumitem} % IRkernel/repr support (it uses the enumerate* environment)
    \usepackage[normalem]{ulem} % ulem is needed to support strikethroughs (\sout)
                                % normalem makes italics be italics, not underlines
    \usepackage{mathrsfs}
    

    
    % Colors for the hyperref package
    \definecolor{urlcolor}{rgb}{0,.145,.698}
    \definecolor{linkcolor}{rgb}{.71,0.21,0.01}
    \definecolor{citecolor}{rgb}{.12,.54,.11}

    % ANSI colors
    \definecolor{ansi-black}{HTML}{3E424D}
    \definecolor{ansi-black-intense}{HTML}{282C36}
    \definecolor{ansi-red}{HTML}{E75C58}
    \definecolor{ansi-red-intense}{HTML}{B22B31}
    \definecolor{ansi-green}{HTML}{00A250}
    \definecolor{ansi-green-intense}{HTML}{007427}
    \definecolor{ansi-yellow}{HTML}{DDB62B}
    \definecolor{ansi-yellow-intense}{HTML}{B27D12}
    \definecolor{ansi-blue}{HTML}{208FFB}
    \definecolor{ansi-blue-intense}{HTML}{0065CA}
    \definecolor{ansi-magenta}{HTML}{D160C4}
    \definecolor{ansi-magenta-intense}{HTML}{A03196}
    \definecolor{ansi-cyan}{HTML}{60C6C8}
    \definecolor{ansi-cyan-intense}{HTML}{258F8F}
    \definecolor{ansi-white}{HTML}{C5C1B4}
    \definecolor{ansi-white-intense}{HTML}{A1A6B2}
    \definecolor{ansi-default-inverse-fg}{HTML}{FFFFFF}
    \definecolor{ansi-default-inverse-bg}{HTML}{000000}

    % common color for the border for error outputs.
    \definecolor{outerrorbackground}{HTML}{FFDFDF}

    % commands and environments needed by pandoc snippets
    % extracted from the output of `pandoc -s`
    \providecommand{\tightlist}{%
      \setlength{\itemsep}{0pt}\setlength{\parskip}{0pt}}
    \DefineVerbatimEnvironment{Highlighting}{Verbatim}{commandchars=\\\{\}}
    % Add ',fontsize=\small' for more characters per line
    \newenvironment{Shaded}{}{}
    \newcommand{\KeywordTok}[1]{\textcolor[rgb]{0.00,0.44,0.13}{\textbf{{#1}}}}
    \newcommand{\DataTypeTok}[1]{\textcolor[rgb]{0.56,0.13,0.00}{{#1}}}
    \newcommand{\DecValTok}[1]{\textcolor[rgb]{0.25,0.63,0.44}{{#1}}}
    \newcommand{\BaseNTok}[1]{\textcolor[rgb]{0.25,0.63,0.44}{{#1}}}
    \newcommand{\FloatTok}[1]{\textcolor[rgb]{0.25,0.63,0.44}{{#1}}}
    \newcommand{\CharTok}[1]{\textcolor[rgb]{0.25,0.44,0.63}{{#1}}}
    \newcommand{\StringTok}[1]{\textcolor[rgb]{0.25,0.44,0.63}{{#1}}}
    \newcommand{\CommentTok}[1]{\textcolor[rgb]{0.38,0.63,0.69}{\textit{{#1}}}}
    \newcommand{\OtherTok}[1]{\textcolor[rgb]{0.00,0.44,0.13}{{#1}}}
    \newcommand{\AlertTok}[1]{\textcolor[rgb]{1.00,0.00,0.00}{\textbf{{#1}}}}
    \newcommand{\FunctionTok}[1]{\textcolor[rgb]{0.02,0.16,0.49}{{#1}}}
    \newcommand{\RegionMarkerTok}[1]{{#1}}
    \newcommand{\ErrorTok}[1]{\textcolor[rgb]{1.00,0.00,0.00}{\textbf{{#1}}}}
    \newcommand{\NormalTok}[1]{{#1}}

    % Additional commands for more recent versions of Pandoc
    \newcommand{\ConstantTok}[1]{\textcolor[rgb]{0.53,0.00,0.00}{{#1}}}
    \newcommand{\SpecialCharTok}[1]{\textcolor[rgb]{0.25,0.44,0.63}{{#1}}}
    \newcommand{\VerbatimStringTok}[1]{\textcolor[rgb]{0.25,0.44,0.63}{{#1}}}
    \newcommand{\SpecialStringTok}[1]{\textcolor[rgb]{0.73,0.40,0.53}{{#1}}}
    \newcommand{\ImportTok}[1]{{#1}}
    \newcommand{\DocumentationTok}[1]{\textcolor[rgb]{0.73,0.13,0.13}{\textit{{#1}}}}
    \newcommand{\AnnotationTok}[1]{\textcolor[rgb]{0.38,0.63,0.69}{\textbf{\textit{{#1}}}}}
    \newcommand{\CommentVarTok}[1]{\textcolor[rgb]{0.38,0.63,0.69}{\textbf{\textit{{#1}}}}}
    \newcommand{\VariableTok}[1]{\textcolor[rgb]{0.10,0.09,0.49}{{#1}}}
    \newcommand{\ControlFlowTok}[1]{\textcolor[rgb]{0.00,0.44,0.13}{\textbf{{#1}}}}
    \newcommand{\OperatorTok}[1]{\textcolor[rgb]{0.40,0.40,0.40}{{#1}}}
    \newcommand{\BuiltInTok}[1]{{#1}}
    \newcommand{\ExtensionTok}[1]{{#1}}
    \newcommand{\PreprocessorTok}[1]{\textcolor[rgb]{0.74,0.48,0.00}{{#1}}}
    \newcommand{\AttributeTok}[1]{\textcolor[rgb]{0.49,0.56,0.16}{{#1}}}
    \newcommand{\InformationTok}[1]{\textcolor[rgb]{0.38,0.63,0.69}{\textbf{\textit{{#1}}}}}
    \newcommand{\WarningTok}[1]{\textcolor[rgb]{0.38,0.63,0.69}{\textbf{\textit{{#1}}}}}


    % Define a nice break command that doesn't care if a line doesn't already
    % exist.
    \def\br{\hspace*{\fill} \\* }
    % Math Jax compatibility definitions
    \def\gt{>}
    \def\lt{<}
    \let\Oldtex\TeX
    \let\Oldlatex\LaTeX
    \renewcommand{\TeX}{\textrm{\Oldtex}}
    \renewcommand{\LaTeX}{\textrm{\Oldlatex}}
    % Document parameters
    % Document title
    \title{Item\_Response\_Theory}
    
    
    
    
    
% Pygments definitions
\makeatletter
\def\PY@reset{\let\PY@it=\relax \let\PY@bf=\relax%
    \let\PY@ul=\relax \let\PY@tc=\relax%
    \let\PY@bc=\relax \let\PY@ff=\relax}
\def\PY@tok#1{\csname PY@tok@#1\endcsname}
\def\PY@toks#1+{\ifx\relax#1\empty\else%
    \PY@tok{#1}\expandafter\PY@toks\fi}
\def\PY@do#1{\PY@bc{\PY@tc{\PY@ul{%
    \PY@it{\PY@bf{\PY@ff{#1}}}}}}}
\def\PY#1#2{\PY@reset\PY@toks#1+\relax+\PY@do{#2}}

\@namedef{PY@tok@w}{\def\PY@tc##1{\textcolor[rgb]{0.73,0.73,0.73}{##1}}}
\@namedef{PY@tok@c}{\let\PY@it=\textit\def\PY@tc##1{\textcolor[rgb]{0.24,0.48,0.48}{##1}}}
\@namedef{PY@tok@cp}{\def\PY@tc##1{\textcolor[rgb]{0.61,0.40,0.00}{##1}}}
\@namedef{PY@tok@k}{\let\PY@bf=\textbf\def\PY@tc##1{\textcolor[rgb]{0.00,0.50,0.00}{##1}}}
\@namedef{PY@tok@kp}{\def\PY@tc##1{\textcolor[rgb]{0.00,0.50,0.00}{##1}}}
\@namedef{PY@tok@kt}{\def\PY@tc##1{\textcolor[rgb]{0.69,0.00,0.25}{##1}}}
\@namedef{PY@tok@o}{\def\PY@tc##1{\textcolor[rgb]{0.40,0.40,0.40}{##1}}}
\@namedef{PY@tok@ow}{\let\PY@bf=\textbf\def\PY@tc##1{\textcolor[rgb]{0.67,0.13,1.00}{##1}}}
\@namedef{PY@tok@nb}{\def\PY@tc##1{\textcolor[rgb]{0.00,0.50,0.00}{##1}}}
\@namedef{PY@tok@nf}{\def\PY@tc##1{\textcolor[rgb]{0.00,0.00,1.00}{##1}}}
\@namedef{PY@tok@nc}{\let\PY@bf=\textbf\def\PY@tc##1{\textcolor[rgb]{0.00,0.00,1.00}{##1}}}
\@namedef{PY@tok@nn}{\let\PY@bf=\textbf\def\PY@tc##1{\textcolor[rgb]{0.00,0.00,1.00}{##1}}}
\@namedef{PY@tok@ne}{\let\PY@bf=\textbf\def\PY@tc##1{\textcolor[rgb]{0.80,0.25,0.22}{##1}}}
\@namedef{PY@tok@nv}{\def\PY@tc##1{\textcolor[rgb]{0.10,0.09,0.49}{##1}}}
\@namedef{PY@tok@no}{\def\PY@tc##1{\textcolor[rgb]{0.53,0.00,0.00}{##1}}}
\@namedef{PY@tok@nl}{\def\PY@tc##1{\textcolor[rgb]{0.46,0.46,0.00}{##1}}}
\@namedef{PY@tok@ni}{\let\PY@bf=\textbf\def\PY@tc##1{\textcolor[rgb]{0.44,0.44,0.44}{##1}}}
\@namedef{PY@tok@na}{\def\PY@tc##1{\textcolor[rgb]{0.41,0.47,0.13}{##1}}}
\@namedef{PY@tok@nt}{\let\PY@bf=\textbf\def\PY@tc##1{\textcolor[rgb]{0.00,0.50,0.00}{##1}}}
\@namedef{PY@tok@nd}{\def\PY@tc##1{\textcolor[rgb]{0.67,0.13,1.00}{##1}}}
\@namedef{PY@tok@s}{\def\PY@tc##1{\textcolor[rgb]{0.73,0.13,0.13}{##1}}}
\@namedef{PY@tok@sd}{\let\PY@it=\textit\def\PY@tc##1{\textcolor[rgb]{0.73,0.13,0.13}{##1}}}
\@namedef{PY@tok@si}{\let\PY@bf=\textbf\def\PY@tc##1{\textcolor[rgb]{0.64,0.35,0.47}{##1}}}
\@namedef{PY@tok@se}{\let\PY@bf=\textbf\def\PY@tc##1{\textcolor[rgb]{0.67,0.36,0.12}{##1}}}
\@namedef{PY@tok@sr}{\def\PY@tc##1{\textcolor[rgb]{0.64,0.35,0.47}{##1}}}
\@namedef{PY@tok@ss}{\def\PY@tc##1{\textcolor[rgb]{0.10,0.09,0.49}{##1}}}
\@namedef{PY@tok@sx}{\def\PY@tc##1{\textcolor[rgb]{0.00,0.50,0.00}{##1}}}
\@namedef{PY@tok@m}{\def\PY@tc##1{\textcolor[rgb]{0.40,0.40,0.40}{##1}}}
\@namedef{PY@tok@gh}{\let\PY@bf=\textbf\def\PY@tc##1{\textcolor[rgb]{0.00,0.00,0.50}{##1}}}
\@namedef{PY@tok@gu}{\let\PY@bf=\textbf\def\PY@tc##1{\textcolor[rgb]{0.50,0.00,0.50}{##1}}}
\@namedef{PY@tok@gd}{\def\PY@tc##1{\textcolor[rgb]{0.63,0.00,0.00}{##1}}}
\@namedef{PY@tok@gi}{\def\PY@tc##1{\textcolor[rgb]{0.00,0.52,0.00}{##1}}}
\@namedef{PY@tok@gr}{\def\PY@tc##1{\textcolor[rgb]{0.89,0.00,0.00}{##1}}}
\@namedef{PY@tok@ge}{\let\PY@it=\textit}
\@namedef{PY@tok@gs}{\let\PY@bf=\textbf}
\@namedef{PY@tok@gp}{\let\PY@bf=\textbf\def\PY@tc##1{\textcolor[rgb]{0.00,0.00,0.50}{##1}}}
\@namedef{PY@tok@go}{\def\PY@tc##1{\textcolor[rgb]{0.44,0.44,0.44}{##1}}}
\@namedef{PY@tok@gt}{\def\PY@tc##1{\textcolor[rgb]{0.00,0.27,0.87}{##1}}}
\@namedef{PY@tok@err}{\def\PY@bc##1{{\setlength{\fboxsep}{\string -\fboxrule}\fcolorbox[rgb]{1.00,0.00,0.00}{1,1,1}{\strut ##1}}}}
\@namedef{PY@tok@kc}{\let\PY@bf=\textbf\def\PY@tc##1{\textcolor[rgb]{0.00,0.50,0.00}{##1}}}
\@namedef{PY@tok@kd}{\let\PY@bf=\textbf\def\PY@tc##1{\textcolor[rgb]{0.00,0.50,0.00}{##1}}}
\@namedef{PY@tok@kn}{\let\PY@bf=\textbf\def\PY@tc##1{\textcolor[rgb]{0.00,0.50,0.00}{##1}}}
\@namedef{PY@tok@kr}{\let\PY@bf=\textbf\def\PY@tc##1{\textcolor[rgb]{0.00,0.50,0.00}{##1}}}
\@namedef{PY@tok@bp}{\def\PY@tc##1{\textcolor[rgb]{0.00,0.50,0.00}{##1}}}
\@namedef{PY@tok@fm}{\def\PY@tc##1{\textcolor[rgb]{0.00,0.00,1.00}{##1}}}
\@namedef{PY@tok@vc}{\def\PY@tc##1{\textcolor[rgb]{0.10,0.09,0.49}{##1}}}
\@namedef{PY@tok@vg}{\def\PY@tc##1{\textcolor[rgb]{0.10,0.09,0.49}{##1}}}
\@namedef{PY@tok@vi}{\def\PY@tc##1{\textcolor[rgb]{0.10,0.09,0.49}{##1}}}
\@namedef{PY@tok@vm}{\def\PY@tc##1{\textcolor[rgb]{0.10,0.09,0.49}{##1}}}
\@namedef{PY@tok@sa}{\def\PY@tc##1{\textcolor[rgb]{0.73,0.13,0.13}{##1}}}
\@namedef{PY@tok@sb}{\def\PY@tc##1{\textcolor[rgb]{0.73,0.13,0.13}{##1}}}
\@namedef{PY@tok@sc}{\def\PY@tc##1{\textcolor[rgb]{0.73,0.13,0.13}{##1}}}
\@namedef{PY@tok@dl}{\def\PY@tc##1{\textcolor[rgb]{0.73,0.13,0.13}{##1}}}
\@namedef{PY@tok@s2}{\def\PY@tc##1{\textcolor[rgb]{0.73,0.13,0.13}{##1}}}
\@namedef{PY@tok@sh}{\def\PY@tc##1{\textcolor[rgb]{0.73,0.13,0.13}{##1}}}
\@namedef{PY@tok@s1}{\def\PY@tc##1{\textcolor[rgb]{0.73,0.13,0.13}{##1}}}
\@namedef{PY@tok@mb}{\def\PY@tc##1{\textcolor[rgb]{0.40,0.40,0.40}{##1}}}
\@namedef{PY@tok@mf}{\def\PY@tc##1{\textcolor[rgb]{0.40,0.40,0.40}{##1}}}
\@namedef{PY@tok@mh}{\def\PY@tc##1{\textcolor[rgb]{0.40,0.40,0.40}{##1}}}
\@namedef{PY@tok@mi}{\def\PY@tc##1{\textcolor[rgb]{0.40,0.40,0.40}{##1}}}
\@namedef{PY@tok@il}{\def\PY@tc##1{\textcolor[rgb]{0.40,0.40,0.40}{##1}}}
\@namedef{PY@tok@mo}{\def\PY@tc##1{\textcolor[rgb]{0.40,0.40,0.40}{##1}}}
\@namedef{PY@tok@ch}{\let\PY@it=\textit\def\PY@tc##1{\textcolor[rgb]{0.24,0.48,0.48}{##1}}}
\@namedef{PY@tok@cm}{\let\PY@it=\textit\def\PY@tc##1{\textcolor[rgb]{0.24,0.48,0.48}{##1}}}
\@namedef{PY@tok@cpf}{\let\PY@it=\textit\def\PY@tc##1{\textcolor[rgb]{0.24,0.48,0.48}{##1}}}
\@namedef{PY@tok@c1}{\let\PY@it=\textit\def\PY@tc##1{\textcolor[rgb]{0.24,0.48,0.48}{##1}}}
\@namedef{PY@tok@cs}{\let\PY@it=\textit\def\PY@tc##1{\textcolor[rgb]{0.24,0.48,0.48}{##1}}}

\def\PYZbs{\char`\\}
\def\PYZus{\char`\_}
\def\PYZob{\char`\{}
\def\PYZcb{\char`\}}
\def\PYZca{\char`\^}
\def\PYZam{\char`\&}
\def\PYZlt{\char`\<}
\def\PYZgt{\char`\>}
\def\PYZsh{\char`\#}
\def\PYZpc{\char`\%}
\def\PYZdl{\char`\$}
\def\PYZhy{\char`\-}
\def\PYZsq{\char`\'}
\def\PYZdq{\char`\"}
\def\PYZti{\char`\~}
% for compatibility with earlier versions
\def\PYZat{@}
\def\PYZlb{[}
\def\PYZrb{]}
\makeatother


    % For linebreaks inside Verbatim environment from package fancyvrb.
    \makeatletter
        \newbox\Wrappedcontinuationbox
        \newbox\Wrappedvisiblespacebox
        \newcommand*\Wrappedvisiblespace {\textcolor{red}{\textvisiblespace}}
        \newcommand*\Wrappedcontinuationsymbol {\textcolor{red}{\llap{\tiny$\m@th\hookrightarrow$}}}
        \newcommand*\Wrappedcontinuationindent {3ex }
        \newcommand*\Wrappedafterbreak {\kern\Wrappedcontinuationindent\copy\Wrappedcontinuationbox}
        % Take advantage of the already applied Pygments mark-up to insert
        % potential linebreaks for TeX processing.
        %        {, <, #, %, $, ' and ": go to next line.
        %        _, }, ^, &, >, - and ~: stay at end of broken line.
        % Use of \textquotesingle for straight quote.
        \newcommand*\Wrappedbreaksatspecials {%
            \def\PYGZus{\discretionary{\char`\_}{\Wrappedafterbreak}{\char`\_}}%
            \def\PYGZob{\discretionary{}{\Wrappedafterbreak\char`\{}{\char`\{}}%
            \def\PYGZcb{\discretionary{\char`\}}{\Wrappedafterbreak}{\char`\}}}%
            \def\PYGZca{\discretionary{\char`\^}{\Wrappedafterbreak}{\char`\^}}%
            \def\PYGZam{\discretionary{\char`\&}{\Wrappedafterbreak}{\char`\&}}%
            \def\PYGZlt{\discretionary{}{\Wrappedafterbreak\char`\<}{\char`\<}}%
            \def\PYGZgt{\discretionary{\char`\>}{\Wrappedafterbreak}{\char`\>}}%
            \def\PYGZsh{\discretionary{}{\Wrappedafterbreak\char`\#}{\char`\#}}%
            \def\PYGZpc{\discretionary{}{\Wrappedafterbreak\char`\%}{\char`\%}}%
            \def\PYGZdl{\discretionary{}{\Wrappedafterbreak\char`\$}{\char`\$}}%
            \def\PYGZhy{\discretionary{\char`\-}{\Wrappedafterbreak}{\char`\-}}%
            \def\PYGZsq{\discretionary{}{\Wrappedafterbreak\textquotesingle}{\textquotesingle}}%
            \def\PYGZdq{\discretionary{}{\Wrappedafterbreak\char`\"}{\char`\"}}%
            \def\PYGZti{\discretionary{\char`\~}{\Wrappedafterbreak}{\char`\~}}%
        }
        % Some characters . , ; ? ! / are not pygmentized.
        % This macro makes them "active" and they will insert potential linebreaks
        \newcommand*\Wrappedbreaksatpunct {%
            \lccode`\~`\.\lowercase{\def~}{\discretionary{\hbox{\char`\.}}{\Wrappedafterbreak}{\hbox{\char`\.}}}%
            \lccode`\~`\,\lowercase{\def~}{\discretionary{\hbox{\char`\,}}{\Wrappedafterbreak}{\hbox{\char`\,}}}%
            \lccode`\~`\;\lowercase{\def~}{\discretionary{\hbox{\char`\;}}{\Wrappedafterbreak}{\hbox{\char`\;}}}%
            \lccode`\~`\:\lowercase{\def~}{\discretionary{\hbox{\char`\:}}{\Wrappedafterbreak}{\hbox{\char`\:}}}%
            \lccode`\~`\?\lowercase{\def~}{\discretionary{\hbox{\char`\?}}{\Wrappedafterbreak}{\hbox{\char`\?}}}%
            \lccode`\~`\!\lowercase{\def~}{\discretionary{\hbox{\char`\!}}{\Wrappedafterbreak}{\hbox{\char`\!}}}%
            \lccode`\~`\/\lowercase{\def~}{\discretionary{\hbox{\char`\/}}{\Wrappedafterbreak}{\hbox{\char`\/}}}%
            \catcode`\.\active
            \catcode`\,\active
            \catcode`\;\active
            \catcode`\:\active
            \catcode`\?\active
            \catcode`\!\active
            \catcode`\/\active
            \lccode`\~`\~
        }
    \makeatother

    \let\OriginalVerbatim=\Verbatim
    \makeatletter
    \renewcommand{\Verbatim}[1][1]{%
        %\parskip\z@skip
        \sbox\Wrappedcontinuationbox {\Wrappedcontinuationsymbol}%
        \sbox\Wrappedvisiblespacebox {\FV@SetupFont\Wrappedvisiblespace}%
        \def\FancyVerbFormatLine ##1{\hsize\linewidth
            \vtop{\raggedright\hyphenpenalty\z@\exhyphenpenalty\z@
                \doublehyphendemerits\z@\finalhyphendemerits\z@
                \strut ##1\strut}%
        }%
        % If the linebreak is at a space, the latter will be displayed as visible
        % space at end of first line, and a continuation symbol starts next line.
        % Stretch/shrink are however usually zero for typewriter font.
        \def\FV@Space {%
            \nobreak\hskip\z@ plus\fontdimen3\font minus\fontdimen4\font
            \discretionary{\copy\Wrappedvisiblespacebox}{\Wrappedafterbreak}
            {\kern\fontdimen2\font}%
        }%

        % Allow breaks at special characters using \PYG... macros.
        \Wrappedbreaksatspecials
        % Breaks at punctuation characters . , ; ? ! and / need catcode=\active
        \OriginalVerbatim[#1,codes*=\Wrappedbreaksatpunct]%
    }
    \makeatother

    % Exact colors from NB
    \definecolor{incolor}{HTML}{303F9F}
    \definecolor{outcolor}{HTML}{D84315}
    \definecolor{cellborder}{HTML}{CFCFCF}
    \definecolor{cellbackground}{HTML}{F7F7F7}

    % prompt
    \makeatletter
    \newcommand{\boxspacing}{\kern\kvtcb@left@rule\kern\kvtcb@boxsep}
    \makeatother
    \newcommand{\prompt}[4]{
        {\ttfamily\llap{{\color{#2}[#3]:\hspace{3pt}#4}}\vspace{-\baselineskip}}
    }
    

    
    % Prevent overflowing lines due to hard-to-break entities
    \sloppy
    % Setup hyperref package
    \hypersetup{
      breaklinks=true,  % so long urls are correctly broken across lines
      colorlinks=true,
      urlcolor=urlcolor,
      linkcolor=linkcolor,
      citecolor=citecolor,
      }
    % Slightly bigger margins than the latex defaults
    
    \geometry{verbose,tmargin=1in,bmargin=1in,lmargin=1in,rmargin=1in}
    
    

\begin{document}
    
    \maketitle
    
    

    
    \section{\texorpdfstring{Detailed Question Analysis: \emph{Item Response
Theory
(IRT)}}{Detailed Question Analysis: Item Response Theory (IRT)}}\label{detailed-question-analysis-item-response-theory-irt}

\begin{itemize}
\tightlist
\item
  IRT models the complex interplay between individual test takers and
  question items
\item
  Probability of a correct resopnse is a function of a latent ``DQL
  ability'' variable (Θ) and each question's \emph{discrimination} and
  \emph{difficulty}
\item
  Some items are more informative than others, and at different levels
  of test taker latent ability
\item
  Similarities with factor analysis \& logit/probit models
\end{itemize}

    \subsubsection{Load libraries, read
data}\label{load-libraries-read-data}

    \begin{tcolorbox}[breakable, size=fbox, boxrule=1pt, pad at break*=1mm,colback=cellbackground, colframe=cellborder]
\prompt{In}{incolor}{152}{\boxspacing}
\begin{Verbatim}[commandchars=\\\{\}]
\PY{n+nf}{options}\PY{p}{(}\PY{n}{warn}\PY{+w}{ }\PY{o}{=}\PY{+w}{ }\PY{l+m}{\PYZhy{}1}\PY{p}{)}\PY{+w}{  }\PY{c+c1}{\PYZsh{}warnings clutter presentation, normally not a good practice}
\PY{n+nf}{library}\PY{p}{(}\PY{n}{ggplot2}\PY{p}{)}
\PY{n+nf}{options}\PY{p}{(}\PY{n}{repr.plot.width}\PY{o}{=}\PY{l+m}{12}\PY{p}{,}\PY{+w}{ }\PY{n}{repr.plot.height}\PY{o}{=}\PY{l+m}{8}\PY{p}{)}

\PY{n+nf}{library}\PY{p}{(}\PY{n}{mirt}\PY{p}{)}\PY{+w}{  }\PY{c+c1}{\PYZsh{} one of several excellent libraries in R for doing IRT}
\PY{n+nf}{library}\PY{p}{(}\PY{n}{ggmirt}\PY{p}{)}\PY{+w}{ }\PY{c+c1}{\PYZsh{} extension to mirt that allows ggplot2 visualization of models}
\PY{c+c1}{\PYZsh{}Read csv file with question cores as correct/wrong (1/0) integers}
\PY{n}{qscores}\PY{+w}{ }\PY{o}{\PYZlt{}\PYZhy{}}\PY{+w}{ }\PY{n+nf}{read.csv}\PY{p}{(}\PY{l+s}{\PYZdq{}}\PY{l+s}{C:/Users/brian\PYZus{}local/PS\PYZus{}demo/qscores.csv\PYZdq{}}\PY{p}{,}\PY{+w}{ }\PY{n}{colClasses}\PY{+w}{ }\PY{o}{=}\PY{+w}{ }\PY{n+nf}{c}\PY{p}{(}\PY{l+s}{\PYZdq{}}\PY{l+s}{integer\PYZdq{}}\PY{p}{)}\PY{p}{)}
\end{Verbatim}
\end{tcolorbox}

    \subsubsection{Fit the IRT Model: 2PL, difficulty (a) and discrimination
(b), but no guessing, because guessing is presumably impossible
here}\label{fit-the-irt-model-2pl-difficulty-a-and-discrimination-b-but-no-guessing-because-guessing-is-presumably-impossible-here}

    \begin{tcolorbox}[breakable, size=fbox, boxrule=1pt, pad at break*=1mm,colback=cellbackground, colframe=cellborder]
\prompt{In}{incolor}{153}{\boxspacing}
\begin{Verbatim}[commandchars=\\\{\}]
\PY{n}{unimodel}\PY{+w}{ }\PY{o}{\PYZlt{}\PYZhy{}}\PY{+w}{ }\PY{l+s}{\PYZsq{}}\PY{l+s}{F1 = 1\PYZhy{}13\PYZsq{}}
\PY{n}{fit2PL}\PY{+w}{ }\PY{o}{\PYZlt{}\PYZhy{}}\PY{+w}{ }\PY{n+nf}{mirt}\PY{p}{(}\PY{n}{data}\PY{+w}{ }\PY{o}{=}\PY{+w}{ }\PY{n}{qscores}\PY{p}{,}\PY{+w}{ }
\PY{+w}{               }\PY{n}{model}\PY{+w}{ }\PY{o}{=}\PY{+w}{ }\PY{n}{unimodel}\PY{p}{,}\PY{+w}{  }\PY{c+c1}{\PYZsh{} Explain what that means here eventually}
\PY{+w}{               }\PY{n}{itemtype}\PY{+w}{ }\PY{o}{=}\PY{+w}{ }\PY{l+s}{\PYZdq{}}\PY{l+s}{2PL\PYZdq{}}\PY{p}{,}\PY{+w}{ }
\PY{+w}{               }\PY{n}{verbose}\PY{+w}{ }\PY{o}{=}\PY{+w}{ }\PY{k+kc}{FALSE}\PY{p}{)}
\end{Verbatim}
\end{tcolorbox}

    \subsubsection{Factor loadings (F1) meausre strength of relationship
between item factors and posited latent factor; h2 is F1\^{}2 and
represents variance in each item accounted for by latent
factor}\label{factor-loadings-f1-meausre-strength-of-relationship-between-item-factors-and-posited-latent-factor-h2-is-f12-and-represents-variance-in-each-item-accounted-for-by-latent-factor}

Relationships are all strong, which is good in general, but implies high
correlation among items, which implies redundant questions and makes
model estimation more difficult

    \begin{tcolorbox}[breakable, size=fbox, boxrule=1pt, pad at break*=1mm,colback=cellbackground, colframe=cellborder]
\prompt{In}{incolor}{154}{\boxspacing}
\begin{Verbatim}[commandchars=\\\{\}]
\PY{n+nf}{summary}\PY{p}{(}\PY{n}{fit2PL}\PY{p}{)}
\end{Verbatim}
\end{tcolorbox}

    \begin{Verbatim}[commandchars=\\\{\}]
       F1    h2
Q2  0.978 0.956
Q3  0.990 0.981
Q4  0.975 0.951
Q5  0.964 0.930
Q6  0.972 0.944
Q7  0.967 0.936
Q8  0.961 0.924
Q9  0.988 0.977
Q10 0.997 0.994
Q11 0.966 0.933
Q12 0.979 0.959
Q13 0.901 0.812
Q14 0.946 0.896

SS loadings:  12.192
Proportion Var:  0.938

Factor correlations:

   F1
F1  1
    \end{Verbatim}

    \subsection{Main Event: IRT parameters}\label{main-event-irt-parameters}

Presents differentiation levels (a), where steep slope = better
differentiation, and difficulty (b), which shows theta level that
corresponds with a .50 probability of correct response. (Guessing --
\emph{g} -- is pre-specified to be zero because if takers need to write
code and grading is all-or-none, then guessing is essentially
impossible.)

    \begin{tcolorbox}[breakable, size=fbox, boxrule=1pt, pad at break*=1mm,colback=cellbackground, colframe=cellborder]
\prompt{In}{incolor}{155}{\boxspacing}
\begin{Verbatim}[commandchars=\\\{\}]
\PY{n}{params2PL}\PY{+w}{ }\PY{o}{\PYZlt{}\PYZhy{}}\PY{+w}{ }\PY{n+nf}{coef}\PY{p}{(}\PY{n}{fit2PL}\PY{p}{,}\PY{+w}{ }\PY{n}{IRTpars}\PY{+w}{ }\PY{o}{=}\PY{+w}{ }\PY{k+kc}{TRUE}\PY{p}{,}\PY{+w}{ }\PY{n}{simplify}\PY{+w}{ }\PY{o}{=}\PY{+w}{ }\PY{k+kc}{TRUE}\PY{p}{)}
\PY{n+nf}{round}\PY{p}{(}\PY{n}{params2PL}\PY{o}{\PYZdl{}}\PY{n}{items}\PY{p}{,}\PY{+w}{ }\PY{l+m}{2}\PY{p}{)}
\end{Verbatim}
\end{tcolorbox}

    A matrix: 13 × 4 of type dbl
\begin{tabular}{r|llll}
  & a & b & g & u\\
\hline
	Q2 &  7.98 & -1.55 & 0 & 1\\
	Q3 & 12.14 & -1.49 & 0 & 1\\
	Q4 &  7.50 & -1.53 & 0 & 1\\
	Q5 &  6.21 & -1.44 & 0 & 1\\
	Q6 &  6.98 & -1.14 & 0 & 1\\
	Q7 &  6.49 & -1.09 & 0 & 1\\
	Q8 &  5.92 & -1.12 & 0 & 1\\
	Q9 & 11.06 & -1.24 & 0 & 1\\
	Q10 & 22.58 & -1.21 & 0 & 1\\
	Q11 &  6.34 & -1.11 & 0 & 1\\
	Q12 &  8.24 & -1.12 & 0 & 1\\
	Q13 &  3.54 & -0.41 & 0 & 1\\
	Q14 &  4.99 & -0.83 & 0 & 1\\
\end{tabular}


    
    \section{Evaluation: Evidence of Model and Item
fit}\label{evaluation-evidence-of-model-and-item-fit}

\subsection{Model Fit}\label{model-fit}

Model fit- Evidence of issues, somewhat mixed bag: M2 significant, but
low RMSEA \& TLI, CFI close to 1; suspect it relates to high correlation
among items

    \begin{tcolorbox}[breakable, size=fbox, boxrule=1pt, pad at break*=1mm,colback=cellbackground, colframe=cellborder]
\prompt{In}{incolor}{156}{\boxspacing}
\begin{Verbatim}[commandchars=\\\{\}]
\PY{n+nf}{M2}\PY{p}{(}\PY{n}{fit2PL}\PY{p}{)}
\end{Verbatim}
\end{tcolorbox}

    A data.frame: 1 × 9
\begin{tabular}{r|lllllllll}
  & M2 & df & p & RMSEA & RMSEA\_5 & RMSEA\_95 & SRMSR & TLI & CFI\\
  & <dbl> & <int> & <dbl> & <dbl> & <dbl> & <dbl> & <dbl> & <dbl> & <dbl>\\
\hline
	stats & 158.7335 & 65 & 8.283518e-10 & 0.04721049 & 0.03791812 & 0.0565313 & 0.04890941 & 0.9953241 & 0.9961034\\
\end{tabular}


    
    \subsection{Item Fit}\label{item-fit}

\begin{itemize}
\tightlist
\item
  Item fis a mixed bag as well
\item
  smaller is better for S\_X2
\item
  p(S\_X2) should be \textgreater{} 0.05 (or whatever alpha you preset)
\item
  Q12 and Q6 are dubious
\item
  Q3 couldn't be estimated due to 0 degrees of freedom, probably because
  of a perfect fit (everyone below a certain Θ got Q3 wrong and everyone
  above got it correct). To Do: Verify this {[} {]}
\end{itemize}

    \begin{tcolorbox}[breakable, size=fbox, boxrule=1pt, pad at break*=1mm,colback=cellbackground, colframe=cellborder]
\prompt{In}{incolor}{157}{\boxspacing}
\begin{Verbatim}[commandchars=\\\{\}]
\PY{n+nf}{itemfit}\PY{p}{(}\PY{n}{fit2PL}\PY{p}{)}
\end{Verbatim}
\end{tcolorbox}

    A mirt\_df: 13 × 5
\begin{tabular}{lllll}
 item & S\_X2 & df.S\_X2 & RMSEA.S\_X2 & p.S\_X2\\
 <chr> & <dbl> & <dbl> & <dbl> & <dbl>\\
\hline
	 Q2  &  0.9389302 & 1 & 0.000000000 & 0.332553125\\
	 Q3  &        NaN & 0 &         NaN &         NaN\\
	 Q4  &  0.5645302 & 2 & 0.000000000 & 0.754073749\\
	 Q5  &  7.6091125 & 4 & 0.037343763 & 0.106993030\\
	 Q6  & 12.9000941 & 5 & 0.049417297 & 0.024333200\\
	 Q7  &  5.0581533 & 6 & 0.000000000 & 0.536376482\\
	 Q8  & 21.8915886 & 6 & 0.063981717 & 0.001266878\\
	 Q9  &  5.1460628 & 3 & 0.033251286 & 0.161408869\\
	 Q10 &  3.6751748 & 1 & 0.064301920 & 0.055228464\\
	 Q11 &  6.0868390 & 6 & 0.004729658 & 0.413533018\\
	 Q12 &  7.8216451 & 4 & 0.038427581 & 0.098334425\\
	 Q13 &  3.2961047 & 3 & 0.012351220 & 0.348185171\\
	 Q14 &  4.1589612 & 5 & 0.000000000 & 0.526763978\\
\end{tabular}


    
    \section{Visualizing Item and Test
Characteristics}\label{visualizing-item-and-test-characteristics}

    \textbf{Note that on all subsequent plots Qn = Item n-1}

    \begin{tcolorbox}[breakable, size=fbox, boxrule=1pt, pad at break*=1mm,colback=cellbackground, colframe=cellborder]
\prompt{In}{incolor}{158}{\boxspacing}
\begin{Verbatim}[commandchars=\\\{\}]
\PY{n+nf}{options}\PY{p}{(}\PY{n}{repr.plot.width}\PY{o}{=}\PY{l+m}{12}\PY{p}{,}\PY{+w}{ }\PY{n}{repr.plot.height}\PY{o}{=}\PY{l+m}{10}\PY{p}{)}
\PY{n+nf}{tracePlot}\PY{p}{(}\PY{n}{fit2PL}\PY{p}{)}\PY{+w}{ }\PY{o}{+}\PY{+w}{ }\PY{n+nf}{theme\PYZus{}minimal}\PY{p}{(}\PY{n}{base\PYZus{}size}\PY{+w}{ }\PY{o}{=}\PY{+w}{ }\PY{l+m}{22}\PY{p}{)}\PY{o}{+}\PY{+w}{ }\PY{n+nf}{geom\PYZus{}line}\PY{p}{(}\PY{n}{size}\PY{+w}{ }\PY{o}{=}\PY{+w}{ }\PY{l+m}{1}\PY{p}{)}
\end{Verbatim}
\end{tcolorbox}

    \begin{center}
    \adjustimage{max size={0.9\linewidth}{0.9\paperheight}}{Item_Response_Theory_files/Item_Response_Theory_15_0.png}
    \end{center}
    { \hspace*{\fill} \\}
    
    \begin{tcolorbox}[breakable, size=fbox, boxrule=1pt, pad at break*=1mm,colback=cellbackground, colframe=cellborder]
\prompt{In}{incolor}{159}{\boxspacing}
\begin{Verbatim}[commandchars=\\\{\}]
\PY{n+nf}{tracePlot}\PY{p}{(}\PY{n}{fit2PL}\PY{p}{,}\PY{+w}{ }\PY{n}{facet}\PY{+w}{ }\PY{o}{=}\PY{+w}{ }\PY{n+nb+bp}{F}\PY{p}{,}\PY{+w}{ }\PY{n}{legend}\PY{+w}{ }\PY{o}{=}\PY{+w}{ }\PY{n+nb+bp}{T}\PY{p}{)}\PY{o}{+}\PY{+w}{ }\PY{n+nf}{scale\PYZus{}color\PYZus{}brewer}\PY{p}{(}\PY{n}{palette}\PY{+w}{ }\PY{o}{=}\PY{+w}{ }\PY{l+s}{\PYZdq{}}\PY{l+s}{Set3\PYZdq{}}\PY{p}{)}\PY{+w}{ }\PY{o}{+}\PY{+w}{ }\PY{n+nf}{theme\PYZus{}minimal}\PY{p}{(}\PY{n}{base\PYZus{}size}\PY{+w}{ }\PY{o}{=}\PY{+w}{ }\PY{l+m}{24}\PY{p}{)}\PY{o}{+}\PY{+w}{ }\PY{n+nf}{geom\PYZus{}line}\PY{p}{(}\PY{n}{size}\PY{+w}{ }\PY{o}{=}\PY{+w}{ }\PY{l+m}{1.5}\PY{p}{)}
\end{Verbatim}
\end{tcolorbox}

    \begin{Verbatim}[commandchars=\\\{\}]
Scale for \textcolor{ansi-green}{colour} is already present.
Adding another scale for \textcolor{ansi-green}{colour}, which will replace the existing
scale.
    \end{Verbatim}

    \begin{center}
    \adjustimage{max size={0.9\linewidth}{0.9\paperheight}}{Item_Response_Theory_files/Item_Response_Theory_16_1.png}
    \end{center}
    { \hspace*{\fill} \\}
    
    \begin{tcolorbox}[breakable, size=fbox, boxrule=1pt, pad at break*=1mm,colback=cellbackground, colframe=cellborder]
\prompt{In}{incolor}{160}{\boxspacing}
\begin{Verbatim}[commandchars=\\\{\}]
\PY{n+nf}{options}\PY{p}{(}\PY{n}{repr.plot.width}\PY{o}{=}\PY{l+m}{12}\PY{p}{,}\PY{+w}{ }\PY{n}{repr.plot.height}\PY{o}{=}\PY{l+m}{12}\PY{p}{)}
\PY{n+nf}{itemInfoPlot}\PY{p}{(}\PY{n}{fit2PL}\PY{p}{,}\PY{+w}{ }\PY{n}{facet}\PY{+w}{ }\PY{o}{=}\PY{+w}{ }\PY{n+nb+bp}{T}\PY{p}{)}\PY{o}{+}\PY{+w}{ }\PY{n+nf}{theme\PYZus{}minimal}\PY{p}{(}\PY{n}{base\PYZus{}size}\PY{+w}{ }\PY{o}{=}\PY{+w}{ }\PY{l+m}{20}\PY{p}{)}\PY{+w}{ }\PY{o}{+}\PY{+w}{ }\PY{n+nf}{geom\PYZus{}line}\PY{p}{(}\PY{n}{size}\PY{+w}{ }\PY{o}{=}\PY{+w}{ }\PY{l+m}{1}\PY{p}{)}
\end{Verbatim}
\end{tcolorbox}

    \begin{center}
    \adjustimage{max size={0.9\linewidth}{0.9\paperheight}}{Item_Response_Theory_files/Item_Response_Theory_17_0.png}
    \end{center}
    { \hspace*{\fill} \\}
    
    \section{Information Plots}\label{information-plots}

Item quality can also be expressed by representing the statistical
\emph{information} (I) an item provides. Higher information
--\textgreater{} more accurate score estimates. The information a given
question provides varies by student DQL ability (Θ)

\textbf{Note again that Qn = Item n-1}

    \begin{tcolorbox}[breakable, size=fbox, boxrule=1pt, pad at break*=1mm,colback=cellbackground, colframe=cellborder]
\prompt{In}{incolor}{161}{\boxspacing}
\begin{Verbatim}[commandchars=\\\{\}]
\PY{n+nf}{options}\PY{p}{(}\PY{n}{repr.plot.width}\PY{o}{=}\PY{l+m}{12}\PY{p}{,}\PY{+w}{ }\PY{n}{repr.plot.height}\PY{o}{=}\PY{l+m}{6}\PY{p}{)}
\PY{n+nf}{itemInfoPlot}\PY{p}{(}\PY{n}{fit2PL}\PY{p}{)}\PY{+w}{ }\PY{o}{+}\PY{+w}{ }\PY{n+nf}{scale\PYZus{}color\PYZus{}brewer}\PY{p}{(}\PY{n}{palette}\PY{+w}{ }\PY{o}{=}\PY{+w}{ }\PY{l+s}{\PYZdq{}}\PY{l+s}{Set3\PYZdq{}}\PY{p}{)}\PY{+w}{ }\PY{o}{+}\PY{+w}{ }\PY{n+nf}{theme\PYZus{}minimal}\PY{p}{(}\PY{n}{base\PYZus{}size}\PY{+w}{ }\PY{o}{=}\PY{+w}{ }\PY{l+m}{24}\PY{p}{)}\PY{o}{+}\PY{+w}{ }\PY{n+nf}{geom\PYZus{}line}\PY{p}{(}\PY{n}{size}\PY{+w}{ }\PY{o}{=}\PY{+w}{ }\PY{l+m}{1.5}\PY{p}{)}
\end{Verbatim}
\end{tcolorbox}

    \begin{Verbatim}[commandchars=\\\{\}]
Scale for \textcolor{ansi-green}{colour} is already present.
Adding another scale for \textcolor{ansi-green}{colour}, which will replace the existing
scale.
    \end{Verbatim}

    \begin{center}
    \adjustimage{max size={0.9\linewidth}{0.9\paperheight}}{Item_Response_Theory_files/Item_Response_Theory_19_1.png}
    \end{center}
    { \hspace*{\fill} \\}
    
    \begin{tcolorbox}[breakable, size=fbox, boxrule=1pt, pad at break*=1mm,colback=cellbackground, colframe=cellborder]
\prompt{In}{incolor}{162}{\boxspacing}
\begin{Verbatim}[commandchars=\\\{\}]
\PY{n+nf}{options}\PY{p}{(}\PY{n}{repr.plot.width}\PY{o}{=}\PY{l+m}{8}\PY{p}{,}\PY{+w}{ }\PY{n}{repr.plot.height}\PY{o}{=}\PY{l+m}{4}\PY{p}{)}
\PY{n+nf}{testInfoPlot}\PY{p}{(}\PY{n}{fit2PL}\PY{p}{,}\PY{+w}{ }\PY{n}{adj\PYZus{}factor}\PY{+w}{ }\PY{o}{=}\PY{+w}{ }\PY{l+m}{2}\PY{p}{)}\PY{+w}{ }\PY{o}{+}\PY{+w}{ }\PY{n+nf}{theme\PYZus{}minimal}\PY{p}{(}\PY{n}{base\PYZus{}size}\PY{+w}{ }\PY{o}{=}\PY{+w}{ }\PY{l+m}{20}\PY{p}{)}
\end{Verbatim}
\end{tcolorbox}

    \begin{center}
    \adjustimage{max size={0.9\linewidth}{0.9\paperheight}}{Item_Response_Theory_files/Item_Response_Theory_20_0.png}
    \end{center}
    { \hspace*{\fill} \\}
    
    \section{Bottom Line Takeaways}\label{bottom-line-takeaways}

\begin{itemize}
\tightlist
\item
  Test appears on the whole excellent at delineating test takers, but
  \emph{almost exclusively at a relatively low level of overall DQL
  ability}
\item
  Test is less good at distinguishing at high levels of performance
\item
  Per IRT, this is a reflection of both the test and its takers
  simultaneously
\item
  If this is a certification test or certification test prep, where
  questions are externally imposed, these results are outstanding
\item
  If this is a home-grown test, this analysis indicates a real
  opportunity to improve ability measurement across full spectrum of
  expected test taker ability
\item
  80/20 rule again: much of this is apparent from a histogram of test
  scores, albeit much more informally \& intuitively
\end{itemize}

    \section{Areas for Future
Investigation}\label{areas-for-future-investigation}

\begin{itemize}
\tightlist
\item
  Explore alternative IRT models (eg Rasch)
\item
  Comparisons to Classical Test Theory (CTT)
\item
  Specify a multidimentional model that measures multiple latent ability
  factors (maybe per tags in original data)
\end{itemize}


    % Add a bibliography block to the postdoc
    
    
    
\end{document}
